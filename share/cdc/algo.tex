
\section{Presentation langage algo et equivalence C}

	Le langage algo est un langage inventé dans le but de faire
comprendre les cours d'algo aux élèves de prépa à l'EPITA, ce qui n'est pas
chose aisée. Ce langage a donc intérêt à offrir moins de souplesse que le C
et à bien faire la distinction entre les différents types, les procédures les
fonctions, etc. Nous allons donc faire un bref aperçu de la 
structure générale de ce langage.

\subsection{structure de base}
\noindent de base le langage se structure de la sorte :

\begin{lstlisting}[style=base]
@[procedure ou fonction]@ µalgorithmeµ [nom algo]
	[declaration des variables et des parametres]
µdebutµ
	[implementation]
µfin algorithmeµ [nom algo]\end{lstlisting}
qui pourrait alors se traduire en C par :
\begin{lstlisting}[style=base]
@[type de retour]@ [nom algo]{[definition des parametres]){
	[declaration des variables]
	[implementation]
}
\end{lstlisting}

On voit déjà apparaître que le langage algo est plus éxigeant en terme de mots
clefs que le C. Passons maintenant à la déclaration des variables et paramètres.

\subsection{Déclaration des variables et des paramètres}
\noindent Parmis les déclarations des variables et des paramètres il existe
plusieurs déclarations possible suivant la composante algorithmique utilisée.

\begin{lstlisting}[style=base]
	µconstantesµ
		id_const = valeur
	µtypesµ
		declaration de type
	µvariablesµ
		@id_type@ id_var 1
\end{lstlisting}
\newpage
qui pourrait alors se traduire en C par : 

\begin{itemize}
	\item[-]\noindent pour les constantes : 
		\begin{lstlisting}[style=base]	
		@const [type]@ id = valeur;\end{lstlisting}	
	\item[-] pour les types :
		\begin{lstlisting}[style=base]			
	 @typedef [type] @id = [type];\end{lstlisting}
	\item[-] pour les variables :
		\begin{lstlisting}[style=base]			
		@[type]@ id = valeur;\end{lstlisting}
\end{itemize}

\subsection{Structure de choix}

Passons maintenant à la presentation des structures de choix et à leurs
équivalences en C. La structure est un peu verbeuse mais sa traduction vers le C
ne pose pose pas de problème majeur.\\
Algo:
\begin{lstlisting}[style=base]
	µsiµ [condition] µalorsµ
		[instruction 1]
	µsinonµ 
		[instruction 2]
	µfin siµ
\end{lstlisting}

\noindent C :

\begin{lstlisting}[style=base]	
	if([condition]){
		[instruction 1]
	}
	else{
		[instruction 2]
	}
\end{lstlisting}

\subsection{Les boucles}

Comme en C, en algo il existe deux types de boucles : \texttt{tant que} et
\texttt{pour}, respectivement \texttt{while} et \texttt{for} en C. Les deux
boucles sont là aussi sensiblement équivalentes dans les deux langages, leurs syntaxes sont :
en algo
\begin{lstlisting}[style=base]
µtant queµ [condition] µfaireµ
	[instruction]
µfin tant queµ
\end{lstlisting}

en C
\begin{lstlisting}[style=base]
while([condition]){

}
\end{lstlisting}

Il existe aussi la boucle \texttt{faire tant que} qui est l'équivalent de la
boucle \texttt{do while} en C :
en Algo
\begin{lstlisting}[style=base]
µfaireµ 
	[instruction]
µtant queµ [condition]
\end{lstlisting}
en C
\begin{lstlisting}[style=base]
do{


}while([condition])
\end{lstlisting}

Et finalement la boucle \texttt{pour} :
en algo
\begin{lstlisting}[style=base]
µpourµ i <- [valeur 1] µjusqu aµ [valeur 2] µfaireµ
	[instruction]
µfin pourµ
\end{lstlisting}

en C
\begin{lstlisting}[style=base]
µforµ([initialisation];[existance];[incrementation]){
	[instruction]
}
\end{lstlisting}


Voilà les bases de la syntaxe du langage algo. Il nous est alors évident que le
passage du langage algo vers un langage plus courant comme le C nous semble
facile, on arrive assez facilement à trouver les correspondances entre les
différentes syntaxes. C'est pour cela que nous avons choisis le C, pour sa
syntaxe simple et par sa grande utilisation.
\\

l'ensemble de document de cour sur le langage peuvent etre trouver sur le
site (si il est à jour) de junior : 
http://algo-td.infoprepa.epita.fr/algo/langage/
